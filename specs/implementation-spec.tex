\documentclass[12pt, a4paper]{article}
\usepackage[a4paper, includeheadfoot, mag=1000, left=2cm, right=1.5cm, top=1.5cm, bottom=1.5cm, headsep=0.8cm, footskip=0.8cm]{geometry}
% Fonts
\usepackage{fontspec, unicode-math}
\setmainfont[Ligatures=TeX]{CMU Serif}
\setmonofont{CMU Typewriter Text}
\usepackage[english, russian]{babel}
% Indent first paragraph
\usepackage{indentfirst}
\setlength{\parskip}{5pt}
% Page headings
\usepackage{fancyhdr}
\pagestyle{fancy}
\renewcommand{\headrulewidth}{0pt}
\setlength{\headheight}{16pt}
\newfontfamily\namefont[Scale=1.2]{Gloria Hallelujah}
\fancyhead{}

\begin{document}

\begin{titlepage}
\begin{center}

\textsc{ФГАОУ ВО «Санкт-Петербургский национальный исследовательский университет информационных технологий, механики и оптики»\\[4mm]
Кафедра вычислительной техники}
\vfill
\textbf{КУРСОВАЯ РАБОТА\\[4mm]
Программирование интернет-приложений\\[4mm]
Третий этап. Уровень бизнес-логики}\\[16mm]
Саржевский Иван Анатольевич
\\[2mm]Лабушев Тимофей Михайлович
\\[2mm]Группа P3202
\vfill
Санкт-Петербург\\[2mm]
2019 г.
\end{center}
\end{titlepage}

\fancyhead[R]{\namefont Sumaju nikki}

\section*{Введение}

Ключевой особенностью разработанной игровой платформы является отсутствие
необходимости в действиях со стороны пользователя для развития игры.

Под развитием игры подразумевается изменение базовых характеристик персонажа —
уровеня и показателя здоровья — а также ресурсов, привязанных к персонажу.

Выделяются следующие виды ресурсов, за обновлением которых может следить пользователь:

\begin{itemize}
\item \textit{посещения уроков}, необходимые для перехода на следующий уровень
\item \textit{навыки обращения с существами}, полученные в бою с ними
\item \textit{заклинания}, отражающие боевые качества персонажа и его способность 
  выстраивать взамоотношения с другими персонажами
\item \textit{взаимоотношения} с персонажами, на создание и поддержание которых
  пользователи влияют напрямую
\item \textit{совы}, используемые пользователем для влияния на персонажа
\end{itemize}

Игровой процесс разделен на дискретные фазы и происходит пошагово. По окончании фазы
происходит обновление характеристик и ресурсов, затем устанавливается следующая фаза
и ее длительность.

На протяжении фазы пользователю демонстрируется \textit{заметка} о текущих действиях
персонажа. Некоторые заметки сохраняются в \textit{дневник}, который также считается
обновляемым ресурсом, отражающим развитие игры.

Каждая заметка ассоциируется с определенной фазой (а также с определенным уроком
для фазы посещения урока, определенным клубом для фазы посещения клуба,
определенным существом для фазы боя.

Заметки предлагаются пользователями. Понравившиеся заметки могут быть отмечены
<<сердечком>>, количество которых отображается публично и служит для поощрения
талантливых создателей.

\section*{Процесс обновления}

Для каждого персонажа доступна информация о его текущей фазе и времени перехода к
следующей. Периодически база данных опрашивается о персонажах с истекшим
временем фазы, и для каждого из них вызывается обработчик. Несколько персонажей
могут обрабатываться одновременно. Данная логика реализуется актором
\texttt{actors.GameProgressionActor}.

Обработчиком является сервис \texttt{services.GameProgressionService},
который определяет все возможные переходы между фазами и их эффекты, делегируя
конкретные обновления специализированным сервисам и объектам доступа к данным \textit{DAO}.
Обработка выполняется в пределах одной транзакции, что предотвращает
частичные изменения.

\section*{Синхронизация обновлений с клиентом}

Для своевременного отражения обновлений между клиентом и сервером устанавливается
двусторонний обмен данными через протокол WebSockets. Между идентификатором пользователя
и его соединениями устанавливается связь, что позволяет отправлять сообщения нескольким активным
устройствам одного пользователя. Данный функционал реализован актором \texttt{actors.SocketMessengerActor}.

Сообщение о смене фазы формируется сервисом \texttt{services.StageService} и содержит:
\begin{itemize}
\item новые характеристики персонажа
\item заметку о новой фазе, показываемую пользователю
\item длительность фазы и прошедшее время, позволяющее рассчитать прогресс хода
\item идентификаторы измененных ресурсов
\end{itemize}

Сравнивая идентификаторы измененных ресурсов, клиент может запросить новые данные
путем обращения к REST API платформы.

\section*{Влияние пользователя на игровой процесс}

В ходе игры пользователю становятся доступны \textit{совы}, обладающие различными эффектами.
Выделяется два вида сов: применяемые немедленно (\textit{immediate}) и участвующие в подсчете
определенных фаз (\textit{non-immediate}).

\subsection*{Совы, применяемые немедленно}

Cовы данного вида позволяют реализовать любое воздействие на текущее состояние
персонажа, поскольку они реализуются как отдельные классы со следующим контрактом:

\begin{verbatim}
abstract class Owl {
  def apply(studentId: Long, payload: JsValue): Either[String, String]
}
\end{verbatim}

\texttt{studentId} является одновременно идентификатором пользователя и его персонажа,
\texttt{payload} является JSON значением, принятым от клиента. Возвращаемое значение —
сообщение об успешном применении или ошибке, показываемое пользователю.

Классы \texttt{Owl} могут принимать DAO как зависимости. Они инстанциируются и вызываются
сервисом \texttt{services.OwlService}.

Ограничением \textit{immediate} сов является то, что они не могут быть учтены в процессе
обновления фазы, так как принимают параметры от клиента и модифицируют глобальное состояние
персонажа.

\subsection*{Фазовые совы}

Для реализации различных бонусов во время подсчета результата фазы применяются совы, у которых
отсутствует конкретная имплементация. Они передаются логике обновления как уникальные строковые идентификаторы
для использования в условных выражениях.

\end{document}
