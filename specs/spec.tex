\documentclass[12pt, a4paper]{article}
\usepackage[a4paper, includeheadfoot, mag=1000, left=2cm, right=1.5cm, top=1.5cm, bottom=1.5cm, headsep=0.8cm, footskip=0.8cm]{geometry}
% Fonts
\usepackage{fontspec, unicode-math}
\setmainfont[Ligatures=TeX]{CMU Serif}
\setmonofont{CMU Typewriter Text}
\usepackage[english, russian]{babel}
% Indent first paragraph
\usepackage{indentfirst}
\setlength{\parskip}{5pt}
% Page headings
\usepackage{fancyhdr}
\pagestyle{fancy}
\renewcommand{\headrulewidth}{0pt}
\setlength{\headheight}{16pt}
\newfontfamily\namefont[Scale=1.2]{Gloria Hallelujah}
\fancyhead{}

\begin{document}

% Title page
\begin{titlepage}
\begin{center}

\textsc{ФГАОУ ВО «Санкт-Петербургский национальный исследовательский университет информационных технологий, механики и оптики»\\[4mm]
Кафедра вычислительной техники}
\vfill
\textbf{КУРСОВАЯ РАБОТА\\[4mm]
Программирование интернет-приложений}\\[16mm]
Саржевский Иван Анатольевич
\\[2mm]Лабушев Тимофей Михайлович
\\[2mm]Группа P3202
\vfill
Санкт-Петербург\\[2mm]
2018 г.

\end{center}
\end{titlepage}


% Table of contents
\begin{center}
{\namefont\huge Sumaju nikki}\\[6mm]
\textsc{Браузерная многопользовательская текстовая игра\\
жанра Zero-Player Game}
\vspace{4mm}
\end{center}
\tableofcontents
\newpage

% Contents
\fancyhead[R]{\namefont Sumaju nikki}

\section{Введение}

Предметом разработки является платформа для многопользовательской текстовой игры
жанра \textit{Zero-Player Game (ZPG)}, использующая браузерный интерфейс и
сохраняющая прогресс игрока на игровом сервере.

В основе жанра лежит концепция протекания игрового процесса
без вмешательства со стороны пользователя. При этом игроку предоставляются
возможности ускорения своего игрового прогресса и
взаимодействия с другими пользователями.

Предполагается, что взаимодействие с платформой будет представлять собой
не длинные сессии, свойственные традиционным играм, а короткие, но частые
посещения, характерные для социальных сетей.

\subsection{Цель создания}

Платформа разрабатывается с целью предоставления творческим молодым людям
возможности самовыражения в игровой форме. Платформа предоставляет набор
базовых игровых механик, в то время как содержательная часть создается
самими пользователями, путём предоставления возможности формировать
предложения для контента

\subsection{Целевая аудитория}

Творческие молодые люди, в возрасте 18-25 лет, знакомые с современными медиа,
активные пользователи интернета. Не имеют времени/желания проводить много
времени за игрой в компьютерные игры, ищут более подходящего под свой стиль
жизни формата.

\section{Требования к аппаратно-программному обеспечению}

\subsection{Требования к серверному обеспечению}

Система, которая обеспечивает выполнение программных продуктов сервера приложений
и хранение данных платформы, должна отвечать следующим требованиям:

\begin{itemize}
\item Поддержка операционной системой бинарного интерфейса приложений (ABI) Linux
\item Наличие сервера баз данных PostgreSQL версии 9.6 и выше
\end{itemize}

\subsection{Требования к клиентскому обеспечению}

Браузерный интерфейс разрабатывается с учетом следующих требований к
программному обеспечению на стороне пользователя:

\begin{itemize}
\item веб-браузер Google Chrome версии 67 и выше или
Mozilla Firefox версии 61 и выше c включенным интерпретатором сценариев JavaScript,
\item отсутствие запрета веб-страницам платформы доступа к внешним ресурсам,
а именно изображениям, шрифтам, таблицам стилей CSS и сценариям JavaScript,
в том числе блокировщиками рекламы.
\end{itemize}

\section{Требования к архитектуре системы}

\subsection{Глоссарий}

\textbf{Уровень back-end} — серверное приложение, с которым взаимодействует
пользовательский интерфейс игры. Уровень back-end обеспечивает хранение данных,
расчет игровых процессов, взаимодействие между пользователями.

\textbf{Уровень front-end} — браузерное приложение, которое реализует
пользовательский интерфейс игры. Уровень front-end включает в себя
HTML-страницы, сценарии JavaScript, таблицы стилей CSS.

\subsection{Уровень back-end}

\begin{enumerate}
\item Серверное приложение должно разрабатываться на платформе \textit{JVM}
с использованием фреймворков \textit{Akka} и \textit{Play}
\item Взаимодействие между уровнями front-end и back-end должно осуществляться
посредством REST API; возможно использование протокола WebSockets для реализации
обновлений в режиме реального времени
\item Серверное приложение должно реализовывать аутентификацию пользователей
с поддержкой входа через социальные сети (OAuth), используя библиотеку
\textit{play-silhouette}. Пароли пользователей должны храниться как
криптографический хэш
\item Серверное приложение должно публиковать push-уведомления об игровых событиях,
используя сервис \textit{Firebase Cloud Messaging}
\item Серверное приложение должно отправлять пользователям еженедельное новостноe
сообщение электронной почты, используя \textit{JavaMail API}
\end{enumerate}

\subsection{Уровень front-end}

\begin{enumerate}
\item Клиентское приложение должно разрабатываться с использованием фреймворка Vue.js
\item Веб-интерфейс должен быть адаптирован для отображения в трех режимах:
\textit{десктопном} (ширина экрана больше 1107px),
\textit{планшетном} (больше 889px) и \textit{мобильном} (меньше 889px)
\item Веб-интерфейс должен предоставлять пользователю возможность подписаться на
push-уведомления
\item Веб-интерфейс должен распознаваться в мобильной ОС Android как
приложение, используя \textit{Web App Manifest}
\end{enumerate}

\end{document}
